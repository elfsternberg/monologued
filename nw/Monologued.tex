% ===> this file was generated automatically by noweave --- better not edit it

% -*- Mode: noweb; noweb-code-mode: rust-mode ; noweb-doc-mode: latex-mode -*-
\documentclass{article}
\usepackage{noweb}
\usepackage{hyperref}
\usepackage{listings}
\usepackage{fontspec, xunicode, xltxtra}
\setmainfont{Centaur MT Std}

\begin{document}

% Generate code and documentation with:
%
% noweave -filter l2h -delay -x -html backbonestore.nw | htmltoc > backbonestore.html
% notangle -Rstore.js backbonestore.nw > store.js
% notangle -Rindex.html backbonestore.nw > index.html

\section{Introduction}

\textbf{Monologued is an RFC-1288 server.}

Monologued is an RFC-1288 (RUIP: Remote User Information Protocol)
server with a restrict (Q1 only) syntax.  It has been mostly written as
an exercise in Rust to learn the MIO (Metal I/O) and libc interfaces, as
well as Rust in general.

RUIP offers two syntaces:

\begin{lstlisting}
{Q1}    ::= [{W}|{W}{S}{U}]{C}
{Q2}    ::= [{W}{S}][{U}]{H}{C}
{U}     ::= username
{H}     ::= @hostname | @hostname{H}
{W}     ::= /W 
{S}     ::= <SP> | <SP>{S}
{C}     ::= <CRLF>
\end{lstlisting}

Monologued supports only the Q1 query, and can be restricted further to
limit the Q1 query only to those users in a defined group, or even to a
list of usernames specified in a file or on the command line.  It
delivers only a user's `.plan` file.  Monologued keeps a small, command
line-limited collection of plans in a cache, to further support
displaying them on request.

In the spirit of traditional IP protocol rules, Monologued is fairly
liberal in its treatment of \{S\} and \{C\}, attempting to make sense of
them.  It's input buffer is limited to 1KB by default; the maximum
output size of a .plan file is 128KB.  Changing these limits would
require a recompile.

\subsection{Literate Program}

A note: this article was written with the
\nwanchorto{http://en.wikipedia.org/wiki/Literate_programming}{Literate
  Programming} toolkit
\nwanchorto{http://www.cs.tufts.edu/~nr/noweb/}{Noweb}. Where you see
something that looks like \texttt{<<this>>}, it's a placeholder for code
described elsewhere in the document.  Placeholders with an equal sign
at the end of them indicate the place where that code is defined.  The
link (U->) indicates that the code you're seeing is used later in the
document, and (<-U) indicates it was used earlier but is being defined
here.

\subsection{Revision}

This is version 0.4 of \textbf{Monologued}.  

Instructions on generating the code and documentation from this document
are provided in the accompanying Makefile.

\subsection{Organization}

An RUIP server is a fairly straightforward programming exercise.  As
we're using MIO, I expect to be using the {\Tt{}select/poll\nwendquote} feature with
which to interact with the outside world, and {\Tt{}libc\nwendquote}'s {\Tt{}getpwnam{\_}r\nwendquote},
{\Tt{}getpwent{\_}r\nwendquote}, and {\Tt{}lstat\nwendquote} functionality to determine the presence
and availability of the user's {\Tt{}.plan\nwendquote} file.

Search: To that end, the basic functionality will be, "For a given
username, return either the plan or a message about discovery failure."
This is straight-up railway-oriented programming.

Cache: Given a username, either return the cached value or run the
search.

Server: given a valid search string, ask the cache for a plan.

Monologued is written in Rust.

\section{The file finder}

An RUIP program is essentially a weak web server; it serves only one
document, understands only one syntax, and that's about it.  There are
some security issues around serving that document, so we'll address
those in this section.

The document we are looking for is, in Unix terminology,
\texttt{\$HOME/.plan}.


\end{document}


\nwfilename{nw/Monologued.nw}
